% $Header: /Users/joseph/Documents/LaTeX/beamer/solutions/generic-talks/generic-ornate-15min-45min.de.tex,v 90e850259b8b 2007/01/28 20:48:30 tantau $

\documentclass{beamer}

% Diese Datei enthält eine Lösungsvorlage für:


% - Vorträge über ein beliebiges Thema.
% - Vortragslänge zwischen 15 und 45 Minuten. 
% - Aussehen des Vortrags ist verschnörkelt/dekorativ.



% Copyright 2004 by Till Tantau <tantau@users.sourceforge.net>.
%
% In principle, this file can be redistributed and/or modified under
% the terms of the GNU Public License, version 2.
%
% However, this file is supposed to be a template to be modified
% for your own needs. For this reason, if you use this file as a
% template and not specifically distribute it as part of a another
% package/program, I grant the extra permission to freely copy and
% modify this file as you see fit and even to delete this copyright
% notice. 



\mode<presentation>
{
  \usetheme{Rochester}
  % oder ...
  %\usecolortheme{beaver}
  %\useoutertheme[subsection=false]{miniframes}
  
\setbeamertemplate{footline}
{%
  \leavevmode%
  \hbox{%
		\begin{beamercolorbox}[wd=\paperwidth,ht=2.25ex,dp=1ex,left]{subsection in head/foot}%
    		%\hspace*{2ex}\usebeamerfont{subsection in head/foot}\insertsubsection
  		%\end{beamercolorbox}%
  		%\begin{beamercolorbox}[wd=.333333\paperwidth,ht=2.25ex,dp=1ex,center]{title in head/foot}%
    		%\usebeamerfont{title in head/foot}\insertshorttitle
  		%\end{beamercolorbox}%
  		%\begin{beamercolorbox}[wd=.333333\paperwidth,ht=2.25ex,dp=1ex,right]{date in head/foot}%
    		%\usebeamerfont{date in head/foot}\insertshortdate{}
    		\hspace*{0.96\paperwidth}
   			\insertframenumber{}\hspace*{2ex} 
  		\end{beamercolorbox}
  }%
  \vskip0pt%
}
  
  
  \setbeamercovered{transparent}
  % oder auch nicht
  
  \definecolor{cismetgrey}{rgb}{0.62,0.62,0.62}
  \definecolor{cismetorange}{rgb}{0.98,0.56,0.06}
  \definecolor{nearblack}{rgb}{0.2,0.2,0.2}

\setbeamercolor{section in toc}{fg=black,bg=white}
\setbeamercolor{alerted text}{fg=cismetgrey!50!black}
\setbeamercolor*{palette primary}{fg=nearblack,bg=cismetgrey!30!white}
\setbeamercolor*{palette secondary}{fg=cismetgrey!70!black,bg=gray!15!white}
\setbeamercolor*{palette tertiary}{bg=cismetgrey!80!black,fg=gray!10!white}
\setbeamercolor*{palette quaternary}{fg=cismetgrey,bg=gray!5!white}

\setbeamercolor*{sidebar}{fg=cismetgrey,bg=gray!15!white}

\setbeamercolor*{palette sidebar primary}{fg=cismetgrey!10!black}
\setbeamercolor*{palette sidebar secondary}{fg=white}
\setbeamercolor*{palette sidebar tertiary}{fg=cismetgrey!50!black}
\setbeamercolor*{palette sidebar quaternary}{fg=gray!10!white}

%\setbeamercolor*{titlelike}{parent=palette primary}
\setbeamercolor{titlelike}{parent=palette primary,fg=nearblack}
\setbeamercolor{frametitle}{bg=cismetgrey}
\setbeamercolor{frametitle right}{bg=cismetgrey!60!white}

\setbeamercolor*{separation line}{}
\setbeamercolor*{fine separation line}{}

  
	\setbeamercolor{itemize item}{fg=cismetorange}
	\setbeamercolor{itemize subitem}{fg=cismetorange}
	\setbeamercolor{enumerate item}{fg=cismetorange}
	
	\setbeamertemplate{navigation symbols}{}%remove navigation symbols
	\setbeamertemplate{itemize items}[triangle]
}


\usepackage[german]{babel}
% oder was auch immer

\usepackage[utf8]{inputenc}
% oder was auch immer

\usepackage{times}
\usepackage[T1]{fontenc}
% Oder was auch immer. Zu beachten ist, das Font und Encoding passen
% müssen. Falls T1 nicht funktioniert, kann man versuchen, die Zeile
% mit fontenc zu löschen.


\title % (optional, nur bei langen Titeln nötig)
{Master-Thesis}

\subtitle
{Kolloquium} % (optional)

\author{Gilles Baatz}
% - Der \inst{?} Befehl sollte nur verwendet werden, wenn die Autoren
%   unterschiedlichen Instituten angehören.

\institute[htw saar] % (optional, aber oft nötig)
{
  HTW des Saarlandes \newline Cismet
}  
  


% - Der \inst{?} Befehl sollte nur verwendet werden, wenn die Autoren
%   unterschiedlichen Instituten angehören.
% - Keep it simple, niemand interessiert sich für die genau Adresse.

\date[Fach] % (optional)
{\today}


\subject{Master Praktische Informatik}
% Dies wird lediglich in den PDF Informationskatalog einfügt. Kann gut
% weggelassen werden.


% Falls eine Logodatei namens "university-logo-filename.xxx" vorhanden
% ist, wobei xxx ein von latex bzw. pdflatex lesbares Graphikformat
% ist, so kann man wie folgt ein Logo einfügen:

% \pgfdeclareimage[height=0.5cm]{university-logo}{university-logo-filename}
% \logo{\pgfuseimage{university-logo}}



% Folgendes sollte gelöscht werden, wenn man nicht am Anfang jedes
% Unterabschnitts die Gliederung nochmal sehen möchte.
%\AtBeginSection[]
%{
  %\begin{frame}<beamer>{Gliederung}
   % \tableofcontents[currentsection]
  %\end{frame}
%}


% Falls Aufzählungen immer schrittweise gezeigt werden sollen, kann
% folgendes Kommando benutzt werden:

%\beamerdefaultoverlayspecification{<+->}



\begin{document}

\begin{frame}
  \titlepage
\end{frame}



% Da dies ein Vorlage für beliebige Vorträge ist, lassen sich kaum
% allgemeine Regeln zur Strukturierung angeben. Da die Vorlage für
% einen Vortrag zwischen 15 und 45 Minuten gedacht ist, fährt man aber
% mit folgenden Regeln oft gut.  

% - Es sollte genau zwei oder drei Abschnitte geben (neben der
%   Zusammenfassung). 
% - *Höchstens* drei Unterabschnitte pro Abschnitt.
% - Pro Rahmen sollte man zwischen 30s und 2min reden. Es sollte also
%   15 bis 30 Rahmen geben.



\section{Einleitung}

\begin{frame}{Hintergrund der Arbeit}
  % - Eine Überschrift fasst einen Rahmen verständlich zusammen. Man
  %   muss sie verstehen können, selbst wenn man nicht den Rest des
  %   Rahmens versteht.

  \begin{itemize}
  \item
    Rationalisierung im R102
    \pause
  \item
    Personalabbau
    \pause
  \item
    Gleiches oder besseres Angebot für Kunden
    %Angebot vertiefen  
  \end{itemize}
\end{frame}

\begin{frame}{Aufteilung der Arbeit}
Implementierung der WuNDa Abrechnungsunterstützung mit anschließender Workflowanalyse von externen Benutzern  \pause
\bigskip
  \begin{itemize}
  \item
    Erster Teil:
    \begin{itemize}
    \item Implementierung der Abrechnungsunterstützung
    \end{itemize}
    \pause
  \item
    Zweiter Teil:
    \begin{itemize}
    \item Workflowanalyse von externen Benutzer
    \end{itemize}  
  \end{itemize}
\end{frame}


\section{Implementierung der Abrechnungsunterstützung}


\begin{frame}{Problemstellung}
  Selbstbeziehen von Produkten \pause	  
  \begin{itemize}
  \item Vorteile externe Benutzer: 
  \begin{itemize}
  	\item unabhängig vom R102 \pause
  	\item Rabatt \pause
  \end{itemize}
  \item<4-> Vorteil R102:
  \begin{itemize}
  \item keine Bearbeitungszeit
  \end{itemize}
  \item<5-> Nachteil R102:
  \begin{itemize}
  \item Zeitaufwändiges Erstellen der Abrechnung
  %Folie mit Erstellen evt einfügen
  \end{itemize}
  \end{itemize}
\end{frame}

\begin{frame}{Lösungsansatz}
  \begin{itemize}
  \item
    Automatisierung der Abrechnung über WuNDa
    \pause
  \item
    Grundlage besteht bereits \pause
  \item Protokollierung der bezogenen Produkte  
  \end{itemize}
\end{frame}

\begin{frame}{Wichtigste Erweiterungen}
  \begin{itemize}
  \item
    Gruppieren von Kunden
    \pause
  \item
    Auflistung von Buchungen \pause
  \item Suche von Buchungen \pause
  \item Erstellung der Abrechnung
  \end{itemize}
\end{frame}


\section{Workflowanalyse von externen Benutzern}

\begin{frame}{Workflowanalyse von externen Benutzern}
  \begin{itemize}
  \item
    Motivation des R102:
    \begin{itemize}
    \item Gewinnen von neuen externen Benutzern
    \item Attraktiveres WuNDa durch neue Funktionalitäten 
    \end{itemize}
    \pause
  \item
    Beleuchtung der Projektentwickler \pause
      \item
    Informationen stammen aus gemeinsamen Treffen
    \pause
  \end{itemize}
\end{frame}

\begin{frame}{Projektentwickler}
  \begin{itemize}
  \item
    Organisation vom Bau von Immobilien
    \pause
  \item
    von Suche der Grundstücke \pause
  \item bis zum Fertigstellen der schlüsselfertigen Immobilie  \pause
  \item Verteilt Aufgaben an Dienstleister
  \end{itemize}
\end{frame}

\begin{frame}{Abfragen eines berechtigten Interesses}
  \begin{itemize}
  \item
    Abfrage ehe Einsicht in Baulasten gewährt wird
    \pause
  \item
    Einsicht und Beziehen von Baulasten kommt häufig vor \pause
  \item Interesse an der Automatisierung \pause
  \item Mögliche Umsetzung:
  \begin{itemize}
  \item Vorgeschalteter Dialog \pause
  \item stichprobenartige Kontrolle
  \item Auffinden von Unregelmäßigkeiten mit WuNDa
  \end{itemize}
  \end{itemize}
\end{frame}

\end{document}


